\chapter{\label{ch:dev}Development practices}

As this is a new software project, we invested a significant effort in setting
up a development infrastructure to ensure our work is tracked,
thoroughly and continually tested, and incrementally improved and documented.
To this end, we have adopted best practices for software development used by
successful open source projects \cite{millman2014}.

\section{\label{sec:vc}Version control}

We use Git\footnote{\url{http://git-scm.com}} as our version control system
(VCS) and GitHub\footnote{\url{https://github.com}} as the public hosting
service for our official \texttt{upstream} repository
\url{https://github.com/statlab/permute}.  Git allows us to track and manage
how our code changes over time as well as review all new functionality before
merging it into the main codebase.  To get new code integrated in the official
\texttt{upstream} repository, we use GitHub's \emph{pull request} mechanism.
This enables us to review code before integrating it.  In the following
section, I describe how we automate our testing to generate reports for all
pull request.  This way we can verify that changes to our code don't break
existing functionality.  Once a pull request is reviewed and accepted, it is
then --- and only then --- merged into the \texttt{upstream} repository.

\section{\label{sec:test}Testing}

We use the \texttt{nose} testing framework for automating our testing
procedures.\footnote{\url{https://nose.readthedocs.org}}  This is the standard
testing framework used by the core packages in the scientific Python ecosystem.
Automating the tests allows us to monitor a proxy for code correctness when
making changes as well as simplifying the code review process for new code. 
The \texttt{nose} testing framework simplifies test creation, discovery,
and running. It has an extensive set of plugins to add functionality
for coverage reporting, test annotation, profiling, isolation, as well
as inspecting and testing documentation.

\begin{figure}
  \begin{centering}
    \includegraphics[width=\textwidth]{fig/pull-request-ci.png}\par
  \end{centering}

  \caption{\label{fig:pull-request}Pull request and continuous integration.}
\end{figure}

Our goal is to test every line of code.  For example, not only do we want to
test every function in our package, but if a specific function has internal
logic we want to test each possible path through the function.  Having tested
each line of code increases our confidence in our codebase, but more
importantly provides us assurance that changes we make do not break existing
code.  It also increases our confidence that new code works, which reduces the
friction of accepting contributions.  Currently over 98\% of our code has at
least one test touch it.

We've configured Travis CI\footnote{\url{https://travis-ci.org}} and
\texttt{coveralls}\footnote{\url{https://coveralls.io}} to be automatically
triggered whenever a commit is made to a pull request or the upstream
master directly.  These systems then run the full test suite 
using different versions of our dependencies (e.g., Python 2.7 and 3.4) 
every time a
new commit is made to a repository or pull request.
This means that when you go to review a pull request you can immediately see
a full report of whether the change breaks any of the tests as well as whether
the new code decreases the overall test coverage (see Figure~\ref{fig:pull-request}).

\section{\label{sec:doc}Documentation}

We use Sphinx\footnote{\url{http://sphinx-doc.org}} as our documentation system
and already have good developer documentation and the foundation for
high-quality user documentation. We've used Python docstrings and the NumPy
docstring standard to document all the modules and functions in
\texttt{permute}.\footnote{\url{https://github.com/numpy/numpy/blob/master/doc/HOWTO\_DOCUMENT.rst.txt}}
Using Sphinx and some NumPy extensions, we have a system for autogenerating the
project documentation (as HTML or PDF) using the docstrings as well as
stand-alone text written in a light-weight markdown-like language.  This
enables us to embed references, figures, code that is auto-run during
documentation generation, as well as mathematics using \LaTeX.

\section{\label{sec:release}Release management}

Our development workflow ensures that the official \texttt{upstream} repository
is always pristine and ready for use.  This means anyone can get our official
upstream master at any point, install it and start using it.  We also 
make official releases with ``binary'' packages\footnote{Presently our code is pure
Python, but we release Python wheels.}
uploaded to the Python Package Index, or PyPI, with release announcements posted
to our mailing list.
To install the latest release of \texttt{permute} and its dependencies, type
the following command from a Bash prompt (assuming you have Python and a recent version of pip): 

\texttt{\$ pip install permute}
