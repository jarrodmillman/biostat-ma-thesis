\chapter{\label{ch:intro}Introduction}

\texttt{permute} is a Python package for permutation tests and confidence
sets.\footnote{\url{http://statlab.github.io/permute}}
Philip B. Stark, Kellie Ottoboni, and I developed this package over the
last year with most of the work occurring during the last few months.
In this report, I briefly explain the purpose of the package (\S~\ref{ch:intro}), our
development practices (\S~\ref{ch:dev}), the currently available functionality (\S~\ref{ch:func}), and
our immediate and long-term roadmap (\S~\ref{ch:nextsteps}).

\section{Permutation tests and confidence sets}

Permutation tests (sometimes referred to as randomization, re-randomization, or
exact tests) are a nonparametric approach to statististical significance
testing.  In a permutation test, the distribution of the test statistic under
the null hypothesis is obtained (exactly or approximately) by considering or
computing the test statistics of all possible relabelings of the observed data.

For example, imagine you observe 10 coin tosses (e.g., HTTHTHTTTT). Assume 
each trial is independent and identically distributed and let the
test statistic be the number of Hs. Under the null hypothesis that the
coin is unbiased, the Hs and Ts are exchangeable and the expected value of the
test statistic is 5. To find the distribution of this test statistic under the
null, count all possible sequences yielding a test statistic of 0,
1, 2, ..., or 10 Hs in 10 tosses.

Consider another example from regression.  In this case, the observed data are
10 $(x, y)$ pairs, which we wish to fit (using least squares) to a linear model
$y = a + bx + \epsilon$.  To test whether $b = 0$, let's use
$\hat{b}/SE(\hat{b})$ as the test statistic.  Under the null hypothesis, there
is no linear relation between $x$ and $y$. Hence the distribution of the test
statistic is found by computing it on all possible $(x, y)$ formed by permuting
the $y$ values among the $x$ values.

From the distribution of the test statistic under the null, you compute
p-values by taking the ratio of the count of the \emph{as extreme} or
\emph{more extreme} test statistics to the total number of such test
statistics.  For instance, in the first example, the test statistic is 3.
Under the null, the expected test statistic is 5.  Thus the more extreme test
statistics are 0, 1, 2, 3, 8, 9, and 10. So the p-value is
\begin{align*}
\frac{\binom{10}{0} + \binom{10}{1} + \binom{10}{2} + \binom{10}{3} +
   \binom{10}{8} + \binom{10}{9} + \binom{10}{10}}{2^{10}}.
\end{align*}
In the second example, there is no analytic expression; however, the
computational procedure is straightforward.

The advantage of this approach is that it 

However, the obvious disadvatage is the computational cost.

What does permute aim to provide

Test statistics in each stratum ...

Methods of combining tests across strata ...

Nonparametric combinations of tests ...


\section{Python}

Python is a high-level, general purpose programming language, which has
become increasingly popular for scientific computing
\cite{millman2011python, Perez2011}. Unlike most high-level languages
used in scientific computing, Python was not specifically designed for
scientific applications.  However, it quickly attracted interest among
scientists and engineers.  Initially, it was employed primarily as a ``glue''
language to couple together low-level numeric libraries written in C or
Fortran with higher-level scientific application languages such as Matlab.

In addition to standard Python, \texttt{permute} makes heavy use of
of third-party libraries NumPy\footnote{\url{http://numpy.org}} and
SciPy\footnote{\url{http://scipy.org}}.

NumPy is

Mersenne Twister
624-vector of 32-bit integers

SciPy
for optimization functions such as Brentq and statistics functions.

Similar to R but not as widely used by statisticians.

Recently has become more attractive for statistical applications due to

Pandas

statsmodels

scikit-learn

missing high-quality resampling approaches

we might consider adding bootstrapping like ...
