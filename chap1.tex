\chapter{Introduction}

\section{Overview}

\texttt{permute} is a Python library for permutation tests and confidence sets.

\begin{itemize}
\item Website (including documentation): \url{http://statlab.github.io/permute}
\item Mailing list: \url{http://groups.google.com/group/permute}
\item Source: \url{https://github.com/statlab/permute}
\item Bug reports: \url{https://github.com/statlab/permute/issues}
\end{itemize}

\subsection{Permutation tests and confidence sets}

Permutation tests and confidence sets for a variety of nonparametric testing
and estimation problems, for a variety of randomization designs.
   
Test statistics in each stratum ...

Methods of combining tests across strata ...

Nonparametric combinations of tests ...

\subsection{Python}

\cite{millman2011python, Perez2011}

\texttt{numpy} \footnote{\url{http://numpy.org}}

\texttt{numpy.random}

\texttt{scipy} \footnote{\url{http://scipy.org}}

\texttt{scipy.optimize}

\texttt{scipy.stats}

\section{Development practices}

\cite{millman2014}

\subsection{Version control}

We are using Git\footnote{\url{http://git-scm.com}} as our version control
system and GitHub\footnote{\url{https://github.com}} as the public hosting
service for our official \texttt{upstream} repository.

\subsubsection{Workflow}

Rather than work on the \texttt{master} branch of our repositories,
we have adopted the policy of performing all code changes on
feature branches.

\subsubsection{Pull requests}

To get new code integrated in the official \texttt{upstream} master,
we use GitHub's \emph{pull request} mechanism.

\subsection{Continuous integration}

Travis CI\footnote{\url{https://travis-ci.org}}

\subsubsection{Testing}

We use the \texttt{nose} testing framework.\footnote{\url{https://nose.readthedocs.org}}

\subsubsection{Coverage}

\texttt{coveralls}\footnote{\url{https://coveralls.io}}
