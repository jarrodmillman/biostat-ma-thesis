\chapter{\label{ch:intro}Introduction}

\texttt{permute} is a Python package for permutation tests and confidence
sets.\footnote{\url{http://statlab.github.io/permute}}
Philip B. Stark, Kellie Ottoboni, and I developed this package over the
last year with most of the work occurring during the last few months.
In this report, I briefly explain the purpose of the package (\S~\ref{ch:intro}), our
development practices (\S~\ref{ch:dev}), the currently available functionality (\S~\ref{ch:func}), and
our immediate and long-term roadmap (\S~\ref{ch:nextsteps}).

\section{Permutation tests}

Permutation tests (sometimes referred to as randomization, re-randomization, or
exact tests) are a nonparametric approach to statistical significance testing.
They were first introduced by R. A. Fisher in 1935 \cite{fisher1935design} and
further developed by E.  Pitman  \cite{pitman1937, pitman1938significance}.
After the introduction of the bootstrap, the ideas were extended in the late
1980's by J. Romano \cite{romano1988bootstrap, romano1989bootstrap}.

In a permutation test, the distribution of the test statistic under the null
hypothesis is obtained (exactly or approximately) by computing the test
statistics on all possible, equally likely relabelings of the observed data. 

%For example, imagine you observe 10 coin tosses (e.g., HTTHTHTTTT). Assume 
%each trial is independent and identically distributed and let the
%test statistic be the number of Hs. Under the null hypothesis that the
%coin is unbiased, the Hs and Ts are exchangeable and the expected value of the
%test statistic is 5. To find the distribution of this test statistic under the
%null, count all possible sequences yielding a test statistic of 0,
%1, 2, ..., or 10 Hs in 10 tosses.

\begin{example}[label=exa:cont] Consider the following randomized, controlled
experiment.  You suspect a specific treatment will increase the growth
rate of a certain type of cell.  To test this hypothesis, you clone 100 cells.
Now there are 200 cells composed of 100 pairs of identical clones.  For each
cloned pair you randomly assign one to treatment.  At the end of the treatment,
you measure the growth rate for all the cells.  The null hypothesis is that
this treatment has no effect. If that is true, then it doesn't matter which of
the cloned pair is labeled as having been treated or not.  So you can generate
new hypothetical datasets from the observed data by exchanging the treatment
and control labels for all the cloned pairs independently, which are equally
likely to have occurred under the null.  This yields a total of $2^{100}$ total
datasets including the observed data and all the equally likely hypothetical
datasets that you generated.

Using a standard parametric approach, you might be interested in using either a
two-sample or a one-sample student $t$-test.  If you use the two-sample student
$t$-test, then the test statistic is the mean of the treated group less the
mean of the control group divided by the pooled estimate of the standard
deviation of the difference in means.  If you use the one-sample student
$t$-test, then the test statistic is the mean of the difference between each
cloned cell divided by the standard error of these differences.  In this
experiment, you would likely prefer the one-sample student $t$-test since the
cloned cells are presumably more similar with each other than with another
randomly chosen cell (and thus more readily compared).  On the other hand if
all your cells were identical clones, then you would prefer the two-sample
student $t$-test.

The permutation testing approach can use the same test statistic as you would
use in the parametric setting, but instead of using the $t$-distribution you
would compute the exact distribution of the test statistic by computing it
on all equally possible datasets (i.e., the observed data as well as your
hypothetical generated data).  The permutation p-value is then the proportion
of the \emph{as} or \emph{more extreme} test statistics.
\end{example}

\begin{example} Given $n=10$ observations of two scalars $(x_i, y_i)$ for
$i = 1, 2, \dots, n$, consider the simple linear regression model
$y_i = a + bx_i + \epsilon_i$.  Assume that the $\epsilon_i$'s are
exchangeable.  To test whether $b = 0$, use the least squares estimate of the slope
normalized by its standard error as the test statistic.  Under the null
hypothesis, there is no linear relation between $x$ and $y$, so all possible
pairs $(x_i,y_j)$ for $i, j = 1, 2, \dots, n$ are equally likely.  Hence the
distribution of the test statistic is found by computing it on all possible
pairs formed by permuting the $y$ values among the $x$ values.

From the distribution of the test statistic under the null, you compute
p-values by taking the ratio of the count of the \emph{as extreme} or
\emph{more extreme} test statistics to the total number of such test
statistics. While there is no analytic expression for this problem, the
computational procedure is straightforward.  Further note that this exact
procedure works regardless of the particular test statistic chosen.

A parametric approach to this problem would begin by imposing additional
assumptions on the noise $\epsilon$.  For example, if it is assumed that
$\epsilon$ comes from identical Gaussian distributions with mean 0
independently realized for each observed pair, then the test statistic has a
$t$-distribution with $n-2$ degrees of freedom.  If this additional assumption
holds, then we can read the p-value off a table.  Note that, unlike in the
permutation test, we were only able to calculate the p-value (even with the
additional assumptions) because we happened to be able to derive the
distribution of this specific test statistic.
\end{example}

%For instance, in the first example, the test statistic is 3.  Under the null,
%the expected test statistic is 5.  Thus the more extreme test statistics are
%0, 1, 2, 3, 8, 9, and 10. So the p-value is \begin{align*} \frac{\binom{10}{0}
%+ \binom{10}{1} + \binom{10}{2} + \binom{10}{3} + \binom{10}{8} +
%\binom{10}{9} + \binom{10}{10}}{2^{10}}.  \end{align*}

In summary, the general procedure for computing a p-value using permutation
testing is: 1) carefully formulate the null hypothesis based on the
experimental design and question of interest, 2) from the observed data,
generate all equally likely possible datasets, 3) compute the test statistic
for the observed and generated hypothetical data, and 4) calculate the
proportion of data sets with test statistic at least as extreme as the
observed.  Of course, as the number of observations increases this procedure
may become computationally intractable.  In this case, simulation can often
produce good approximate results.

The advantage of this approach is that it produces exact (or approximately
exact) results using a strong null hypothesis (i.e., we can employ whatever
test statistic we are interested in) and weaker assumptions (since we don't
need to derive an analytic form for the distribution of our chosen test
statistic) about the generating process. Thus permutation testing allows us to
focus on the exact question of interest, rather than a similar question which
is amenable to analysis.  Additionally, the rationale justifying these claims
doesn't require asymptotic theory; this means we have exact results for small
samples. However, the obvious disadvantage is the computational cost.  With
increasing computational resources and improved algorithms this limitation
is less of a problem.

\section{Confidence sets}

Despite the duality between tests and confidence sets, computing the permutation
confidence set often requires additional assumptions.  

\begin{example}[continues=exa:cont] In addition to the above stated
assumptions, you further assume an additive model where the effect of treatment
adds a constant amount $d$ to the control response of the cell.  With this
additional assumption, you can now estimate the treatment effect by inverting
the permutation test.  Under the additive model, you assume that if you remove
the true treatment effect $d$ the growth rates of the treated and untreated
clones would then be similar.  Hence an interval estimate of $d$ can be
constructed by looking at all possible values of $d$ for which the null is not
rejected by the permutation test.  Since you assume the treatment effect is
positive under the alternative, you will want a one-sided confidence interval.
Assuming that the p-value calculated by the permutation test described above is
less than $\alpha$, where $\alpha$ is the predetermined significance level of
your test.  So you would then take the observed test statistic and increase it
by a small amount and perform the permutation again.  You keep repeating this
process until you find the maximum value $d_U$ such that when you remove this
amount from the treated cell's growth rate, you no longer reject the null
hypothesis at significance level $\alpha$.  Then $(0, d_U)$ is a $(1-\alpha)100\%$
permutation confidence interval for $d$.
\end{example}

\section{Python}

Python is a high-level, general purpose programming language, which has become
increasingly popular for scientific computing \cite{millman2011python,
Perez2011}. Unlike some high-level languages used in scientific computing,
Python was not specifically designed for scientific applications.  However, it
quickly attracted interest among scientists and engineers.  Initially, it was
employed primarily as a ``glue'' language to couple together compiled binaries
for scientific applications written in C or Fortran \cite{dubois2007guest}.

As more scientists and engineers began using Python, they started developing
third party libraries to provide additional functionality for scientific and
numeric computing.  In particular, NumPy\footnote{\url{http://numpy.org}},
SciPy\footnote{\url{http://scipy.org}}, and
matplotlib\footnote{\url{http://matplotlib.org}} provide a core foundation on
which other scientific Python packages (such as \texttt{permute}) build. NumPy
provides the basic n-dimensional array data structure and a small core
functionality such as linear algebra routines to compute on this data
structure.  SciPy adds additional general routines on top of this core
functionality necessary for scientific computing including basic statistics and
optimization.  Complementing these data structures and algorithms, matplotlib
provides publication quality 2D plotting.  

While it is beyond the scope of this report to explore these packages in more
detail, I note that the pseudo-random number generator (PRNG) provided by NumPy
is the Mersenne Twister.  The Mersenne Twister is an efficient PRNG with a
sufficiently large period for most statistical simulations.\footnote{It is
the default PRNG in R as well.  While sufficient for most statistical simulations,
for other applications such as cryptography it may be insufficient.}  In
addition to providing a high-quality PRNG, NumPy implements Knuth shuffling ---
an efficient (and simple) algorithm for uniformly generating permutations of
sequences.

While Python is similar to R in many respects and is widely used in scientific
and numerical computing, it lacks R's extensive support for statistical
applications.  Recently, however, it has become more attractive for statistical
applications due to new packages such as Pandas, statsmodels, and
scikit-learn.  Our intention is to help accelerate this trend by supplying a
high-quality, rigorously tested, and statistically sound package for a large
variety of permutation tests and confidence sets. As the package matures, we
anticipate contributing generic functionality upstream to the packages we
depend on as appropriate.

\section{Putting it all together}

Our aim is for \texttt{permute} to help increase the use of permutation methods
as well as increase the usefulness of Python for statistical data analysis.
While we wish to see an increased use of permutation methods, our aim is to not
only make these methods easier to apply, but we also want to help researchers
apply these methods correctly.  To this end, we intend to ultimately provided
many worked examples illustrating the thought process as well as the mechanical
application of these methods.  Further our goal is to provide tools to
prototype and implement custom tests such that researchers can easily develop
permutation tests appropriate for their experiments.  Experimental design needs
to be carefully considered in designing permutation tests. We'd like to provide
the tools that will enable scientists to conduct tests that actually correspond
to their experimental design.  Finally, we aim to provide Python tools to
statisticians, and provide better statistical tools to Python users.
