\chapter{\label{ch:intro}Introduction}

\texttt{permute} is a Python package for permutation tests and confidence
sets.\footnote{\url{http://statlab.github.io/permute}}
Philip B. Stark, Kellie Ottoboni, and I developed this package over the
last year with most of the work occurring during the last few months.
In this report, I briefly explain the purpose of the package (\S~\ref{ch:intro}), our
development practices (\S~\ref{ch:dev}), the currently available functionality (\S~\ref{ch:func}), and
our immediate and long-term roadmap (\S~\ref{ch:nextsteps}).

\section{Permutation tests and confidence sets}

Permutation tests (sometimes referred to as randomization, re-randomization, or
exact tests) are a nonparametric approach to statistical significance
testing.  In a permutation test, the distribution of the test statistic under
the null hypothesis is obtained (exactly or approximately) by 
computing the test statistics of all possible relabelings of the observed data.

%For example, imagine you observe 10 coin tosses (e.g., HTTHTHTTTT). Assume 
%each trial is independent and identically distributed and let the
%test statistic be the number of Hs. Under the null hypothesis that the
%coin is unbiased, the Hs and Ts are exchangeable and the expected value of the
%test statistic is 5. To find the distribution of this test statistic under the
%null, count all possible sequences yielding a test statistic of 0,
%1, 2, ..., or 10 Hs in 10 tosses.

For example, consider the following regression problem.  In this case, the
observed data might be 10 $(x, y)$ pairs, which we wish to fit (using least
squares) to a linear model $y = a + bx + \epsilon$.  To test whether $b = 0$,
let's use $\hat{b}/\text{SE}(\hat{b})$ as the test statistic.  Under the null
hypothesis, there is no linear relation between $x$ and $y$, so all possible
combinations of pairs $(x,y)$ are equally likely.  Hence the distribution of
the test statistic is found by computing it on all possible $(x, y)$ formed by
permuting the $y$ values among the $x$ values.

From the distribution of the test statistic under the null, you compute
p-values by taking the ratio of the count of the \emph{as extreme} or
\emph{more extreme} test statistics to the total number of such test
statistics. While there is no analytic expression for this problem, the
computational procedure is straightforward.  Further note that this exact
procedure works regardless of the particular test statistic choosen.

A parameteric approach to this problem would begin by imposing additional
assumptions on the noise $\epsilon$.  For example, if it is assumed that
$\epsilon$ comes from identical Guassian distributions independently realized
for each observed pair, then the test statistic has a $t$-distribution with
$n-2$ degrees of freedom.  If this additional assumption holds, then we can read
the p-value off a table.  Note that, unlike in the permutation test, we were
only able to calculate the p-value (even with the additional assumptions)
because we happened to be able to derive the distribution of this specific
test statistic. 

%For instance, in the first example, the test statistic is 3.
%Under the null, the expected test statistic is 5.  Thus the more extreme test
%statistics are 0, 1, 2, 3, 8, 9, and 10. So the p-value is
%\begin{align*}
%\frac{\binom{10}{0} + \binom{10}{1} + \binom{10}{2} + \binom{10}{3} +
%   \binom{10}{8} + \binom{10}{9} + \binom{10}{10}}{2^{10}}.
%\end{align*}

In summary, the general procedure for computing a p-value using permutation
testing is: 1) carefully formulate the null hypothesis based on the
experimental design and question of interest, 2) from the observed data,
generate all equally likely possible datasets, 3) compute the test statistic
for the observed and generated hypothetical data, and 4) calculate the
proportion of data sets with test statistic at least as extreme as the
observed.  Of course, as the number of observations increases this procedure
may become computationally intractable.  In this case, simulation can often
produce good approximate results.

The advantage of this approach is that it produces exact (or approximately
exact) results using a strong null hypothesis (i.e., we can employee whatever
test statistic we are interested in) and weaker assumptions (since we don't
need to derive an analytic form for the distribution of our choosen test
statistic) about the generating process. Thus permutation testing allows us to
focus on the exact question of interest, rather than a similiar question which
is amenable to analysis.  Additionally, the rationale justifying these claims
doesn't require asymptotic theory; this means it should be easier for
non-mathematically trained scientists to fully understand.  However, the
obvious disadvantage is the computational cost.

What does permute aim to provide
\begin{itemize}
\item experimental design needs to be carefully considered in designing
  permutation tests. We'd like to provide the tools that will enable scientists
  to conduct tests that actually correspond to their experimental design
\item provide Python tools to statisticians, and provide better statistical
  tools to Python users.
\end{itemize}

\section{Python}

Python is a high-level, general purpose programming language, which has become
increasingly popular for scientific computing \cite{millman2011python,
Perez2011}. Unlike some high-level languages used in scientific computing,
Python was not specifically designed for scientific applications.  However, it
quickly attracted interest among scientists and engineers.  Initially, it was
employed primarily as a ``glue'' language to couple together compiled binaries
for scientific applications written in C or Fortran \cite{dubois2007guest}.

As more scientists and engineers began using Python, they started developing
third party libraries to provide additional functionality for scientific
and numeric computing.  In particular, NumPy\footnote{\url{http://numpy.org}},
SciPy\footnote{\url{http://scipy.org}}, and matplotlib\footnote{
\url{http://matplotlib.org}} provide a core foundation on which other
scientific Python packages (such as \texttt{permute}) build. NumPy
provides the basic n-dimensional array data structure and a small core
functionality such as linear algebra routines to compute on this
data structure.  SciPy adds additional general routines on top
of this core functionality necessary for scientific computing including
basic statistics and optimization.  Complementing these data structures
and algorithms, matplotlib provides publication quality 2D plotting.  

While it is beyond the scope of this report to explore these packages in more
detail, I note that the pseudo-random number generator (PRNG) provided by NumPy
is the Mersenne Twister.  The Mersenne Twister is an efficient PRNG with a
sufficiently large period for most statistical simulations.\footnote{It is
default PRNG in R as well.  While sufficient for most statistical simulations,
for other applications such as cryptography it may be insufficient.}  In
addition to providing a high-quality PRNG, NumPy implements Knuth shuffling ---
an efficient (and simple) algorithm for uniformly generating permutations of
sequences.

While Python is similar to R in many respects and is widely used in scientific
and numerical computing, it lacks R's extensive support for statistical
applications.  Recently, however, it has become more attractive for statistical
applications due to new packages such as Pandas, statsmodels, and
scikit-learn.  Our intention is to help accelerate this trend by supplying a
high-quality, rigorously tested, and statistically sound package for a large
variety of permutation tests and confidence sets. As the package matures, we
anticipate contributing generic functionality upstream to the packages we
depend on as appropriate.
