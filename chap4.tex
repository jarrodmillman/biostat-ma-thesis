\chapter{\label{ch:nextsteps}Next steps}

While the development of \texttt{permute} will continue to be guided by the
scientific projects we engage with, there are a few things already planned.

\section{\label{sec:book}``Permutation Tests for Complex Data'' Examples}

On the companion
website\footnote{\url{http://www.wiley.com/legacy/wileychi/pesarin/material.html}}
to ``Permutation Tests for Complex Data: Theory, Applications and
Software'' by F. Pesarin and L. Salmaso \cite{pesarin2010permutation}, the
authors provide several R functions for implementing the application examples
presented in their book.  They've also made their datasets available for download
(as well as some Matlab and SAS code).  In preparation for the second edition,
we will be implementing these examples in Python as part of our package.

As mentioned above, these datasets are already included in our package.  We have
also created a GitHub
repository,\footnote{\url{https://github.com/statlab/permuter}} where we have
made some tentative improvements to the code.  In order to create a test suite
to ensure we are able to independently produce the results given by the R
implementation, we plan to develop it into an R package.  Since the existing R
code was released for the purpose of illustration, rather than as an attempt by
the authors to provide an R library for reuse, we do not currently plan to
submit the package to CRAN.  However, it will be publicly-available on GitHub
and will be easily installable (we will need to make it easy to install for the
purposes of our automated testing anyway).

In addition to using the examples and R code as part of our testing suite, we
will also include a detailed discussion of each example from the book in our
user documentation.  While our intention is not to further develop and improve
the R package, we will fix bugs and refactor the code as necessary to ensure
that we can validate our results for the examples using \texttt{permute} with
an independent implementation in R.  As part of our documentation we will
include snippets of the R code, in the hopes that it might serve users of
\texttt{permute} who are more familiar with R as they work through the examples
in Python.

\section{Missing features and new projects}

Now that we have our development infrastructure in place and some experience in
designing the API, it will be easier to add new permutation tests (e.g., for
slope in linear regression, for independence and association, or for symmetry).
Even without motivating scientific collaborations, we will begin adding these
tests (and others) and their associated confidence sets as it will help us
refine our naming and call signature conventions.

Our motivating scientific collaborations have so far involved small- to
moderate-sized datasets.  We will soon begin working with larger datasets.  In
particular, we will begin using our software for problems from
genomics and functional neuroimaging.  Both of these problem domains already
make use of permutation tests and the size of the datasets will help us
better understand the performance and scalability issues.

\section{Design decisions}

Thus far we have limited our core dependencies to NumPy and SciPy.
As we add new functionality, we will be tempted to leverage additional
external packages.  For pure Python packages, this will cause little
difficulty.  However, many of the packages we will be interested in
are partially implemented in C, C{}\verb!++!, or Fortran.  Depending
on packages that require compilation, will increase the complexity
of installation for some users.

While choosing what packages to depend on will require some thought, a more
difficult set of questions will arise as we attempt to finalize our application
programming interface (API).  We've already spent a lot of effort unifying our
call signatures and naming conventions.  However, we have begun discussing more
generic structures for specifying test statistics.  For instance,
here is the call signature of our two sample permutation test for the equality
of two means: 
\begin{verbatim}
  def two_sample(x, y, reps=10**5, stat='mean', alternative='greater',
                 keep_dist=False, interval=False, level=0.95, seed=None)
\end{verbatim}
Currently, you can specify the test statistic by setting \texttt{stat} to
\texttt{"mean"} or \texttt{"t"} depending on whether you want to use the
difference in the means or the two-sample t-statistic.  The function checks
which string is passed in and then chooses the correct function for the test
statistic.  Ideally, we would like to allow users to pass a function in
directly rather than a string.  If we go this route it may make sense to
consider creating test statistics classes that we can instantiate with the test
statistic function.  Another possibility would be to see if we can create
something like R's formula for specifying our permutation models.

Finally, as we start implementing more tests and applying our package to larger
datasets, we will need to take performance more seriously.  While we have been
able to stick to a pure Python implementation so far, we will eventual need to
write C extensions or use an NumPy-aware optimizing static compiler for Python such
Cython\footnote{\url{http://cython.org}} or
Numba\footnote{\url{http://numba.pydata.org}} to improve computation time.  As
permutation tests are inherently parallelizable, we will also need to consider
providing parallel processing support.
