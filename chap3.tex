\chapter{\label{ch:func}Current functionality}

While the scope of our package is large, we have initially focused on a few
features based on our current collaborations.  In particular, we have
implemented built-in datasets, functions for data cleaning and quality
assurance, core functionality like various types of permutations and confidence
intervals, some stratified tests and confidence sets, and a novel stratified
permutation test for multirater interrater reliability.\footnote{In
Appendix~\ref{app:ex}, I have included a couple worked examples to give the
reader a sense of how they would use \texttt{permute} in practice.  For
additional documentation and examples, please see the project website:
\url{http://statlab.github.io/permute}.}

\section{Data}

To simplify testing, for use in our documentation, and to provide users with
example data, we provide several built-in datasets. At this point, the majority
of the datasets come from the examples used in ``Permutation Tests for Complex
Data: Theory, Applications and Software'' \cite{pesarin2010permutation}.  These
datasets are discussed in more details in \S~\ref{sec:book}.  We also have a
data set, which we use as an example for the stratified multirater interrater
reliability test (\S~\ref{sec:irr}).

\section{Utility functions}

We provide several utility functions that are useful in multiple specific
permutation tests as well as functions useful for reproducible tests, which
rely on a PRNG.  This includes core functions for permuting various data
structures in different ways.  Currently, we have functions for permuting the
rows of a matrix in-place as well as permuting conditions within each group.

\section{Quality assurance}

As we added datasets, we also incrementally added tools to simplify data
cleaning or quality assurance (QA).  Currently, we provide two helper
functions: one for reporting duplicate rows in a matrix in various formats and
another for reporting duplicate consecutive rows in a matrix in various
formats.  Rather than having a list of planned QA functionality, we have been
opportunistic in only implementing functionality as needed by the datasets
included in our \texttt{data} module.

\section{Core tools}

The \texttt{core} module contains classic (non-stratified) permutation tests
and confidence sets.  For instance we have a function for computing a one- or
two-sided, two-sample permutation test for the equality of two means.  We also
have functions for computing confidence intervals for a binomial.
While one of our collaborators, Anne Boring from OFCE-Sciences Po in Paris, was
visiting to work on a study evaluating bias in student evaluations of teachers
\cite{boring2015}, we added code to simulate permutation p-values for a Spearman
correlation coefficient.

\section{Stratified tests}

This module was the first we developed and contains code adapted from an
IPython notebook written by Philip demonstrating how to perform a stratified
permutation test using the sum of the differences in means between two or more
conditions in each group (stratum) as the test statistic.
We also have a function to simulate permutation p-values for a stratified
Spearman correlation test, which was implemented during Anne Boring's visit.

\section{\label{sec:irr}Interrater reliability}

The multirater interrater reliability module was motivated by a problem
presented to us by Naomi Stark.  She was interested in analyzing rater
reliability in a dataset collected during the following experiment.  Several
different raters watched several videos.  Each video was paused at regular time
intervals and each rater was asked to indicating whether or not a particular
event occurred in the video segment they just viewed.\footnote{In fact, they
were asked whether several different events occurred in the video segment, but
we consider each event separately.  So for the purpose of this report, we will
assume that there is only one event that is either present or absent during
each video segment.} The question posed to us was to determine whether these
responses (or ratings) were reliable between raters. 

For each video and rater, we have a vector of 0s and 1s indicating whether the
event occurred or not during each time interval.  Under the null hypothesis, we
assume that the elements of this vector are exchangeable.  So the permutation
distribution is derived from permuting the elements of each rater's rating
vector independently.  For our test statistic, we count the number of
concordant pairs.  We then calculate the p-value for each video.  Then we
compute an overall p-value by combining the results across video (i.e., strata)
using the nonparametric combination of tests (NPC) described in Pesarin and
Salmaso.

In 1960, Cohen \cite{cohen1960} introduced the kappa statistic for measuring
the agreement between two raters correcting for chance.  Since then a number of
extensions have been proposed, but we were unable to find one that provided a
stratified permutation test for multiple raters. We developed our own based on
a proposal by Philip Stark. The complete details of our method are included in
the documentation and are implemented with tests in our code.  A descriptive
manuscript will soon follow.

%That is, within each stratum, we count the number of concordant pairs of
%assignments.
