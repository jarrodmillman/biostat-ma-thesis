\chapter{\label{ch:func}Current functionality}

While the scope of our package is large, we have initially focused on a
few features.  In particular, we have implemented built-in data sets,
functions for exploratory data analysis, core functionality like various
types of permutations and confidence intervals, some stratified tests
and confidence sets, and a test for interrater reliability.

\section{Data}

To simplify testing as well as providing some standard datasets for
users to experiment with as they learn how our software works, we
have a data module, which provides quick and ready access to a few
built-in data sets. At this point, the majority of the datasets come
from the examples used in ``Permutation tests for complex data: theory,
applications and software'' \cite{pesarin2010permutation}.  These
datasets are discussed in more details in \S~\ref{sec:book}.  We also
have a data set, which we use as an example for the interrater
reliability test (\S~\ref{sec:irr}).

Loading one of the packaged data sets involves importing and then
invoking a helper function, which handles the data loading.  For
example, here is how you would load the data set used for the
interrater reliability test:
\begin{verbatim}
In [1]: from permute.data import nsgk

In [2]: x = nsgk()
\end{verbatim} 

\section{Exploratory data analysis}

As we've added data sets, we've also incrementally added tools to simplify
exploratory data analysis (EDA).  Currently, we provide two helper functions.
One for reporting duplicate rows in a matrix in various formats.  Another for
reporting duplicate consecutive rows in a matrix in various formats.  Rather
than having a list of planned EDA functionality, we have been opportunistic
and have only implemented functionality as needed for data sets we are adding
to the \texttt{data} module.

\section{Core tools}

The \texttt{core} module contains basic functions that could be useful in
multiple specific permutation tests.  

This module provides functions for permuting various data structures in
different ways.  Currently we have functions for permuting the rows of
a matrix in-place as well as permuting conditions within each group.


Compute the confidence interval for a binomial.


Simulate permutation p-value for Spearman correlation coefficient


One-sided or two-sided, two-sample permutation test for equality of two means,
with p-value estimated by simulated random sampling with reps replications.

Tests the hypothesis that x and y are a random partition of x,y against the
alternative that x comes from a population with mean

a. greater than that of the population from which y comes, if side = ‘greater’

b. less than that of the population from which y comes, if side = ‘less’

c. different from that of the population from which y comes, if side = ‘two-sided’

\section{Stratified permutation testing}

Test statistics in each stratum ...

Methods of combining tests across strata ...

Nonparametric combinations of tests ...

\section{\label{sec:irr}Interrater reliability}

The test statistic within stratum $s$ is

\begin{align*}
\rho_s &\equiv \frac{1}{N_s {R \choose 2}} \sum_{i=1}^{N_s}
              \sum_{r=1}^{R-1} \sum_{v=r+1}^R 1(L_{s,i,r} = L_{s,i,v}) \\
       &= \frac{1}{N_s R(R-1)} \sum_{i=1}^{N_s}
                (y_{si}(y_{si}-1) + (R-y_{si})(R-y_{si}-1)).
\end{align*}

That is, within each stratum, we count the number of concordant pairs of
assignments.

